\documentclass[]{uow-exercises}

\usepackage{caption}
\usepackage{subcaption}

\usepackage{minted}
\usepackage[
backgroundcolor = gray!5
, hidealllines=true
]{mdframed}

\surroundwithmdframed{minted}
\usemintedstyle{lovelace}

\author{Michael B. Gale}
\title{CS2D2}

\setminted[haskell]{fontsize=\small}
\setminted[bash]{fontsize=\small}
\newcommand{\haskellIn}[1]{\mintinline[fontsize=\small]{haskell}{#1}}
\newcommand{\bashIn}[1]{\mintinline[fontsize=\small]{bash}{#1}}

\begin{document}

\fancyhead[LE, RO]{\emph{CS2D2 Programming Paradigms}}
\pagestyle{fancy}
\thispagestyle{fancy}

\begin{center}
    \textbf{\LARGE CS2D2 Exercises - Wednesday}
\end{center} \hrule \vspace{0.5cm}

To get started, download the archive with the skeleton codes for today's labs from:
\begin{center}\small
    \url{https://warwick.ac.uk/fac/sci/dcs/teaching/material/cs2d2/haskell.zip}
\end{center}

\paragraph{DCS machines} The best way to complete the exercises is on the DCS machines. To get things set up there, simply run the following script in a terminal:
\begin{minted}{bash}
/modules/cs141/haskell-setup.sh
\end{minted}
This only needs to be done once. Close the terminal and open a new one afterwards.

\paragraph{Own machine} If you prefer to work on your own device, you will need to install a few things. See the installation instructions at
\begin{center}
    \url{https://docs.haskellstack.org/en/stable/README/}
\end{center}
to install the Haskell tool stack. Once installed, run
\begin{minted}{bash}
stack setup --resolver=lts-16.27
\end{minted}

\paragraph{Web} You could also use an online tool such as
\begin{center}
    \url{https://replit.com/languages/haskell}
\end{center}
simply by copying and pasting your code into there, but you will be more limited in what you can do.

\section{Wednesday morning - Functional Programming}

Open Visual Studio Code and open the \texttt{\small lab-fp} folder you have extracted from the \texttt{haskell.zip} archive (File > Add folder to workspace) and then open a terminal window inside of Visual Studio Code (Terminal > New Terminal) which should appear at the bottom. 

A very useful tool which you may find helpful for completing the exercises and debugging your code is a Read-Eval-Print Loop (or REPL for short). This is offered by many modern programming languages and development tools, including the Glasgow Haskell Compiler (which \bashIn{stack} is using behind the scenes). You can launch the REPL by invoking the following command in your terminal while in the \texttt{\small lab-fp} folder:
\begin{minted}{bash}
$ stack repl
\end{minted}
This loads the code for these exercises and allows you to enter arbitrary expressions which the REPL will evaluate for you. You should see something similar to the following prompt:
\begin{minted}{haskell}
*Lab Lab>
\end{minted}
Simply enter an expression like \haskellIn{1+1} and hit enter to evaluate it. The REPL will print the result of evaluating the expression:
\begin{minted}{haskell}
*Lab Lab> 1+1
2
\end{minted}
There is a \texttt{\small src/Lab.hs} file in the \texttt{\small lab-fp} directory which contains some definitions for this lab. Because we have run \bashIn{stack repl} in the directory with the code (\texttt{\small lab-fp}), the REPL has automatically loaded the the \texttt{\small src/Lab.hs} file for us, so you can refer to definitions in it:
\begin{minted}{haskell}
*Lab Lab> name
"Michael"
*Lab Lab> age * 2
60
\end{minted}

\task{Open the \texttt{\small src/Lab.hs} file in your preferred text editor and modify the definitions of \haskellIn{age} and \haskellIn{name} to match your name and age.}

Instead of typing an expression which should be evaluated by the REPL, you may also type in a command (all commands start with a colon). The following commands are supported (among others):
\begin{center}
\begin{tabular}{|l|l|}
\hline 
    \texttt{\small :q}   & Quits the REPL. \\ 
\hline 
    \texttt{\small :r}  & Reloads all files that are currently loaded. \\ 
\hline 
    \texttt{\small :l FILENAME} & Loads \texttt{\small FILENAME} into the REPL. \\
\hline
\end{tabular} 
\end{center}
Assuming you did not close the REPL, it will still be running. The REPL does not automatically check for updates to any files that are currently loaded, so you will have to reload it with the \texttt{\small :r} command. In general, the \texttt{\small :r} command reloads all files that are loaded in the REPL. Now try evaluating \haskellIn{age} and \haskellIn{name} again. They should match whatever values you changed them to.

\task{Complete the definitions of \haskellIn{triple}, \haskellIn{not}, \haskellIn{and}, \haskellIn{max}, and \haskellIn{perimeterRect} in \texttt{\small src/Lab.hs} according to the instructions provided there. You can test them in the REPL to see if you have got them right. Remember to reload the file once you have made some changes.}

\task{The skeleton code for these exercises also comes with a test suite which you can use to test the correctness of your functions. Simply run \texttt{\small stack test} in a terminal window (not in the REPL!) to run all unit tests against your code. At this point, all of the tests for the previous exercises should pass and all the ones for the following exercises should fail. For all failed tests, there will be an explanation of why they failed. If all of the tests for the previous tasks pass, you can proceed. As you progress through the following exercises, you may wish to re-run \texttt{\small stack test} whenever you want to test your answers.}

\taskLine

\task{With the help of the \haskellIn{reverse} function from the standard library, which reverses lists, complete the definition of \haskellIn{isPalindrome} which checks whether a list of characters is a palindrome. For example:}
\begin{minted}{haskell}
isPalindrome "radar" ==> True 
isPalindrome "2020"  ==> False
\end{minted}

\taskLine 

Just like lists are constructed by ``cons''-ing elements to smaller lists, starting with the empty list, lists can be deconstructed through pattern-matching by ``uncons''-ing elements from the start of a list. For example, the \haskellIn{head} function returns the first element of a list and discards the rest:
\begin{minted}{haskell}
head (x:xs) = x
\end{minted}
We can have nested patterns so that \emph{e.g.} the following function will retrieve the second element of a list:
\begin{minted}{haskell}
headOfTail (x:y:xs) = y
\end{minted}
Note that this is not a function you should ever write though, since it will crash for any list that does not have at least two elements. The \haskellIn{head} function will also crash if it is given a list that does not have at least one element.

\task{Complete the definition of \haskellIn{validModuleCode} which should determine whether a given argument is a valid Computer Science module code. For the purpose of this exercise, a valid module code consists of five characters: the letters \haskellIn{'c'} and \haskellIn{'s'}, followed by three digits. You may find the \haskellIn{isDigit} function from \haskellIn{Data.Char} useful. For example:}
\begin{minted}{haskell}
validModuleCode ['c','s','1','4','1'] => True 
validModuleCode ['c','s','2','6','3'] => True
validModuleCode ['c','s','4','0','7'] => True
validModuleCode ['e','s','1','4','1'] => False
validModuleCode ['c','a','k','e','s'] => False
validModuleCode ['l','i','e']   => False
\end{minted}
Note that writing \haskellIn{"cs141"} means the same as \haskellIn{['c','s','1','4','1']} in Haskell.

\taskLine

\task[task:elem-explicit]{Using recursion, complete the definition of the \haskellIn{elem} function which should determine whether some value is contained in a list of values. For example, \haskellIn{elem 4 [4,8,15,4]} should evaluate to \haskellIn{True} and \haskellIn{elem 7 [4,8,15,4]} should evaluate to \haskellIn{False}.}

\task{Using recursion, complete the definition of \haskellIn{maximum} which should find the greatest element of the list given as argument. For example, \haskellIn{maximum [1,2,3,2,1]} should evaluate to \haskellIn{3}. You may assume that \haskellIn{maximum} will never be called with the empty list so you do not need to define an equation for that case.}

\section{Wednesday afternoon - Higher-order programming}

Open Visual Studio Code and open the \texttt{\small lab-hop} folder you have extracted from the \texttt{haskell.zip} archive (File > Add folder to workspace) and then open a terminal window inside of Visual Studio Code (Terminal > New Terminal) which should appear at the bottom. 

The goal of this lab will be to implement \emph{Picross} or nonogram puzzles in Haskell. If you are unfamiliar with these puzzles, you can find an explanation and an example at
\begin{center}
    \url{https://en.wikipedia.org/wiki/Nonogram}
\end{center}
In a nutshell, we have an (initially empty) grid of pixels where the goal of the puzzle is to determine which of these pixels should be filled in and which ones should not be in order to form an image. To help with this, rows and columns are annotated with integers which describe the length of uninterrupted sequences of filled pixels in a given row or column. For example, if a row is annotated with \texttt{\small 2, 4} it means that there is one sequence of two filled pixels and one sequence of four filled pixels in the row -- the two sequences cannot touch. All other pixels in the row are unfilled.

The \texttt{\small src/Picross.hs} file in the \texttt{\small lab-hop} directory contains the definitions we need to complete to implement the game. The REPL (\texttt{\small stack repl}) can be used to test your functions and, once complete, you can run the game with \texttt{\small stack run}.

\taskLine

We represent images as lists of lists of integers. For example, 
\begin{center}
    \haskellIn{[[1,0,1], [0,1,0], [1,0,1]]} 
\end{center}
represents a 3x3 image where the four corners and the middle are filled in. That is, we use \haskellIn{1} to mean ``filled'' and \haskellIn{0} to mean ``unfilled''. We will also later use \haskellIn{-1} to mean a pixel for which we don't yet know whether it is filled or not.

\task{Implement the \haskellIn{makeBlank} function which, given an image (a list of rows), should replace the values of all pixels with \haskellIn{-1}. For example:}
\begin{minted}{haskell}
makeBlank [[1,1,0,1,1,1], [1,0,1,0,1,0]]
 => [[-1,-1,-1,-1,-1,-1], [-1,-1,-1,-1,-1,-1]]
\end{minted}

For all of the following tasks, note that in Haskell a string is just a list of characters. For example, writing \haskellIn{"at"} is the same as \haskellIn{['a', 't']} and can be written interchangeably. Particularly when testing things in the REPL, do not be surprised if you get a result in the string notation, rather than in the list of characters notation.

\task{Implement the \haskellIn{showCell} function which, given an integer representing a width \haskellIn{w} and an integer representing the value of a pixel, should return \haskellIn{w}-many copies of the \haskellIn{symbol} representing the pixel value. For example:}
\begin{minted}{haskell}
showCell 3 1 => ['X', 'X', 'X']
showCell 3 (-1) => ['?', '?', '?']
showCell 2 1 => ['X', 'X']
\end{minted}
Hint: use the \haskellIn{symbol} function, which is already provided for you, to obtain the character corresponding to the pixel value. 

\task{Implement the \haskellIn{showRow} function which, given a width and a row of pixels, should return the results of \haskellIn{showCell} for each pixel in a list. For example:}
\begin{minted}{haskell}
showRow 2 [1,1,(-1)]
 => [['X', 'X'], ['X', 'X'], ['?', '?']]
\end{minted}

\task{Implement the \haskellIn{pad} function which, given an integer representing a length \haskellIn{n} and a string (list of characters) should return that string, padded with whitespaces up to the length \haskellIn{n}. For example:}
\begin{minted}{haskell}
pad 4 "ab" => "ab  "
pad 2 "ab" => "ab"
pad 1 "ab" => "ab"
pad 5 "ab" => "ab   "
\end{minted}

\taskLine

\task{To help in creating nonograms from images, implement the \haskellIn{sequences} function which, given a row, returns a list of the ``hints'' for the row. For example:}
\begin{minted}{haskell}
sequences [1,1,0,1,1,1] => [2,3]
sequences [1,0,1,0,1] => [1,1,1]
sequences [0,0,0] => []
\end{minted}
There are many different approaches you could take to implementing this: 
\begin{itemize}
    \item You could use functions we have covered in the lectures, plus some helper functions you write yourself or from the \haskellIn{Data.List} module in Haskell's standard library: \url{https://hackage.haskell.org/package/base-4.15.0.0/docs/Data-List.html}
    \item You could write a recursive function which does it for you instead.
    \item You could read about list comprehensions and use these to simplify parts of the definition: 
    
    {\small \url{http://learnyouahaskell.com/starting-out#im-a-list-comprehension}}
\end{itemize}

\task{Implement the \haskellIn{rows} function which, given a list of rows, should determine the hints for all the rows and return them in a list. For example:}
\begin{minted}{haskell}
rows [[1,1,0,1,1,1], [1,0,1,0,1,0]]
 => [[2,3], [1,1,1]]
\end{minted}

\task{Implement the \haskellIn{columns} function which, given a list of rows, should determine the hints for all the \emph{columns} and return them in a list. For example:}
\begin{minted}{haskell}
columns [[1,1,0,1,1,1], [1,0,1,0,1,0]]
 => [[2],[1],[1],[1],[2],[1]]
\end{minted}
Hint: you may find the \haskellIn{transpose} function helpful\footnote{\url{https://hackage.haskell.org/package/base-4.15.0.0/docs/Data-List.html\#v:transpose}}.

\taskLine

\task{Implement the \haskellIn{showLabel} function, which given the result of \haskellIn{rows} or \haskellIn{columns}, i.e. a list of integers representing a hint, should format it as a comma-separated string. For example:}
\begin{minted}{haskell}
showLabel [2,3] => "2,3"
showLabel [2] => "2"
showLabel [1,1,1] => "1,1,1"
\end{minted}
Hint: the \haskellIn{show} function can be used to turn an integer into a string and you may find the \haskellIn{intercalate}\footnote{\url{https://hackage.haskell.org/package/base-4.15.0.0/docs/Data-List.html\#v:intercalate}} or \haskellIn{intersperse}\footnote{\url{https://hackage.haskell.org/package/base-4.15.0.0/docs/Data-List.html\#v:intersperse}} functions helpful.

\task{Implement the \haskellIn{showLabels} function which, given a list of hints, should compute the result of \haskellIn{showLabel} for each of them and return the results in a list. For example:}
\begin{minted}{haskell}
showLabels [[2,3], [1,1,1]]
 => ["2,3", "1,1,1"]
\end{minted}

\taskLine

\section*{Extensions}

If you have got this far and still have time left or want to do this in your own time, you could try to write a function which, given the result of \haskellIn{makeBlank} and all the hints for the image, tries to find a solution for the puzzle.

	
\end{document}
